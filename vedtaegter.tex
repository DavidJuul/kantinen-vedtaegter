% vim: spell spelllang=da_dk

\documentclass[a4paper, 10pt]{article}

% Tekstindkodning og orddeling
\usepackage[utf8]{inputenc}
\usepackage[danish, english]{babel}

\title{Vedtægter for Kantineforeningen ved \\ Datalogisk Institut,
Københavns Universitet}

\def\vedtagsdato{den 4. december 2017}

\author{}
\date{Vedtaget \vedtagsdato{}}

% Palatino skrifttype
\usepackage[T1]{fontenc}
\usepackage{mathpazo}
\linespread{1.05}

\renewcommand\thesection{\textsection\arabic{section}}

% Fint hoved og fod
\usepackage{fancyhdr}
\usepackage{lastpage}
\renewcommand{\headrulewidth}{0in}
\renewcommand{\headsep}{40pt}
\setlength{\headheight}{25pt}
\pagestyle{fancy}
\makeatletter
\let\headtitle\@title
\let\headdate\@date
\makeatother
\lhead{\headtitle}
\rhead{\headdate\\}
\cfoot{Side~\thepage~af~\pageref{LastPage}}
\fancypagestyle{first}{%
  \fancyhf{}%
  \cfoot{Side~\thepage~af~\pageref{LastPage}}%
}

\usepackage{enumitem}

\newenvironment{stykenum}{
  \begin{enumerate}[%
    label=Stk.~\arabic*., ref=\textsection~\theenumi~Stk.~\arabic*, start=2]
}{\end{enumerate}}

\begin{document}

\maketitle
\thispagestyle{first}

\section{Navn}

\label{navn} Foreningens navn er `Kantineforening ved Datalogisk
Institut'. Hjemsted er København.

\section{Formål}

\label{formaal} Foreningens formål er at drive marketenderi i de
dertil hørende lokaler i Kø\-benhavns Universitet.

\section{Medlemmer}

\label{medlemmer} Som medlemmer kan optages enhver, som er ansat eller
studerende ved Datalogisk Institut. Medlemskab tegnes for et studieår
af gangen. Medlemskab af kantineforeningen indebærer:

\medskip

\begin{itemize}

\item Ret til at købe varer fra kantinens automater og frokostbuffeten

\item Ret til at opbevare madvarer i kantinens brugerkøleskabe

\item Ret til at tilberede mad i kantinen, såfremt denne ikke er
udlånt til andre formål

\item Ret til at låne kantinen til private formål, såfremt en
ansøgning om lånet er afsendt til bestyrelsen og godkendt.

\item Ret til at deltage i de arrangementer og fester, som bestyrelsen
afholder for brugerne.

\item Ret til at benytte faciliteterne i hyggehjørnet udenfor normal
studietid

\end{itemize}

Der betales ikke kontingent, da foreningens drift skal hvile i sig
selv. Med\-lemmerne er forpligtet til ved deres arbejde at medvirke
til kantinens drift, som beskrevet i ordensreglerne for brug af
kantinen, som forefindes på kantineforeningens hjemmeside.  Ophører et
medlems tilhørsforhold til instituttet, ophører medlemskabet
ligeledes.  Såfremt et medlems opførsel skønnes at være i konflikt med
ordensreglerne kan bestyrelsen (i samråd med instituttets le\-delse)
ophæve dennes medlemskab midlertidigt eller permanent. Dette
forudæstter dog at mindst 2/3 af bestyrelsen stemmer herfor. Dersom
den pågældende ønsker det skal spørgsmålet forelægges den følgende
generalforsamling. Udmeldelse skal ske til bestyrelsen.

\subsection{Regler for brug af kantinen af ikke-medlemmer}

Det kræves at enhver som benytter kantinens tilbud og lokaler
overholder de gældende ordensregler. Såfremt bestyrelsen skønner at
dette ikke har været tilfældet er bestyrelsen berettiget til at
formene personen eller gruppen lov til at benytte kantinen i
fremtiden.

\section{Bestyrelsen}

Foreningens ledes af en bestyrelse, der består af en formand,
kasserer, samt seks til femten menige bestyrelsesmedlemmer.
Bestyrelsen konstituerer sig selv, dvs. fastlægger selv hvilke poster,
der yderligere skal være, samt hvem der skal varetage de forskellige
poster. Bestyre\-lsen er selvsupplerende. Afgøre\-lser inden for
bestyrelsen træffes ved almindeligt flertal. Ved stemmelighed
bortfalder forslaget.

\begin{stykenum}

\item Formand og kasserer, ved ophørt tilhørsforhold til DIKU kan ved
indstemmelse til en (evt. ekstraordinær) generalforsamling forblive
som medlemmer i foreningen i op til 3 måneder efter ophør, og kan
forblive ved sin post i den periode, med henblik på overdragelse af
posterne og fuldmagterne.

\end{stykenum}

\section{Generalforsamling}

\begin{itemize}

\item Foreningens øverste myndighed er generalforsamlingen.

\item Ordinær generalforsamling afholdes medio marts.

\item Indkaldelse til ordinær generalforsamling skal ske med mindst to
ugers varsel, med en uges frist til at indsende forslag til optagelse
på dagsordenen. Forslag, der skal behandles på en generalforsamling
skal stå på dagsordenen. Dagsordenen offentliggøres mindst tre dage
før generalforsamlingen. Indkaldelse og dagsorden offentliggøres både
på de relevante mailinglister og ved opslag i kantinen.

\item Ekstraordinær generalforsamling kan indkaldes, når to
bestyrelsesmed\-lemmer eller 20 af medlemmerne ønsker det. For den
ekstraordinære generalforsamling gælder samme regler som for den
ordinære pånær: Indkaldelse til ekstraordinær generalforsamling skal
ske med mindst fire ugers varsel, med to ugers frist til at indsende
forslag til optagelse på dagsordenen og dagsordenen skal
offentliggøres mindst en uge før generalforsamlingen.

\item Såfremt der er medlemmer der ønsker at indtræde i bestyrelsen,
skal de til den siddende bestyrelse skriftligt give til kende, at de
ønsker at stille op senest en uge efter indkaldelsen.  Hvis dette er
tilfældet går hele bestyrelsen på valg. Kandidaten skal angive
hvilke(n) post(er) han/hun stiller op til (formand, kasserer eller
menigt medlem). Bestyrelsen udarbejder en kandidatliste som
offentliggøres senest samtidig med dagsordenen.

\item Dagsordenen for den ordinære generalforsamling skal mindst
indeholde følgende \\ punkter:

\begin{enumerate}

\item Valg af dirigent

\item Valg af referent

\item Formandens beretning

\item Kasserens beretning

\item Fremlæggelse af regnskab

\item Indkomne forslag

\item Valg af to revisorer

\item Eventuelt

\end{enumerate}

\end{itemize}

\section{Afstemning}

Til generalforsamlingen træffes beslutninger ved simpelt flertal,
hvilket vil sige mere end halvdelen af de afgivne stemmer.
Stemmeberettigede er ethvert med\-lem ved personligt fremmøde.
Beslutninger kan foretages ved håndsoprækning. Hvis nogen ønsker det,
skal der foretages skriftlig afstemning.

\subsection*{Ved skriftlig afstemning}

Blanke stemmer er både gyldige og afgivne stemmer, dvs. tæller med i
antallet af afgivne stemmer. En stemmeseddel er ugyldig:

\begin{itemize}

\item Hvis den indeholder mere end et ja eller nej.

\item Hvis den indeholder andet end navnene på de kandidater der
stemmes om.

\item Hvis den indeholder flere navne end antallet af kandidater, der
skal væl\-ges.

\item Hvis der benyttes andre sedler end de udleverede stemmesedler.

\item Hvis der benyttes tegn eller påskrives bemærkninger, som kan
afsløre, hvem stemmeafgiveren er.

\end{itemize}

Ugyldige stemmer tælles med som afgivne stemmer.

\subsection*{Valg af bestyrelse}

Formand og kasserer vælges direkte af generalforsamlingen ved simpelt
flertal. Er der dog flere end to kandidater og får ingen af disse mere
end halvdelen af de afgivne stemmer, stemmes der igen blandt de to,
der fik flest stemmer i første runde. Formanden vælges først. Dernæst
kassereren, og til sidst de resterende medlemmer. Der udleveres
stemmeseddel med plads til det antal kandidater der skal vælges.
Navnene skal angives i den rækkefølge kandidaterne ønskes valgt.

\subsubsection*{Optælling}

\begin{itemize}

\item Der udregnes fordelingstal svarende til gyldige stemmer/antal poster.

\item Har nogen kandidater opnået fordelingstallet på 1.-stemmer er disse
\\ valgt. Nr. 1 er den med flest 1.-stemmer osv.

\item For de kandidater, der ikke opnår fordelingstallet med
1.-stemmer, ses på 2.-stemmer lagt sammen med 1.-stemmer. Hvis der
stadig er ubesatte poster, ses på 1.-stemmer lagt sammen med 2.stemmer
og 3.-stemmer, osv.

\item Er der stemmelighed på 1.-stemmer, er antallet af 2.-stemmer
afgørende for hvem er nr.  1, osv.

\item Har to kandidater stemmelighed ved alle prioritetsstemmer
foretages \\ lodtrækning.

\end{itemize}

\subsection*{Valg af revisorer}

Valget gælder for det indeværende års regnskaber. Valg foregår efter
samme princip som for bestyrelsen.

\section{Vederlag}

Hvervene som bestyrelsesmedlemmer og revisorer er ulønnede.

\section{Tegning}

Foreningen tegnes ved underskrift af formand, kasserer eller to
bestyrelses\-medlemmer. Bestyrelsen er økonomisk ansvarlig.
Bestyrelsen har pligt til at føre et hensigtsmæssigt regnskab, og
mindst en gang pr. halvår at foretage en opgørelse af
varebeholdningen. Bestyrelsen må påse, at driften hviler i sig selv.
Bestyrelsen fastsætter regler for den daglige drift. Bestyrelsen har
pligt til at sørge for, at de til enhver tid gældende offentlige
forskrifter for marketenderier overholdes. Bestyrelsens medlemmer er
ikke personligt økonomisk ansvar\-lige for eventuelle fordringer imod
foreningen.

\section{Vedtægtsændringer}

Vedtægtsændringer foretages på en generalforsamling med mindst 2/3
flertal (kvalificeret flertal) af de afgivne stemmer.

\begin{stykenum}

\item Vedtægtsændringer skal begrundes skriftligt, og være
bestyrelsen i hænderne mindst en uge før generalforsamlingen.

\item Vedtægtsændringerne træder i kraft umiddelbart efter deres
vedtagelse.

\item I tilfælde af tvetydigheder i vedtægterne undervejs i en
generalforsamling, vedtages tolkning og/eller vedtægtsændring med
henblik på klar\-gørelse af vedtægterne ved kvalificeret flertal.

\end{stykenum}

\section{Opløsning}

\subsection*{Opløsning efter ønske fra foreningens medlemmer}

Foreningen kan opløses på følgende måde: Beslutningen skal foretages
på en generalforsamling og vedtages af mindst 2/3 af samtlige
medlemmer. Er 2/3 af medlemmerne ikke til stede ved
generalforsamlingen, kan bestyrelsen indkalde til en ny
generalforsamling inden fire uger, hvor opløsning kan ske med 2/3 af
de tilstedeværendes stemmer. Ved den sidste generalforsamling
besluttes hvad der skal ske med foreningens midler.

\subsection*{Opløsning som følge af opsigelse af nuværende lokaler,
f.eks. ved flytning}

I tilfælde af at kantinens lokaler bliver opsagt, f.eks. hvis
instituttet flyttes, vil den aktive kantinebestyrelse undersøge
mulighederne for at videreføre alle, eller dele af den nuværende
studenterkantines aktiviteter i andre lokaler, så vidt det er muligt.

Hvis en flytning af kantinen vurderes mulig, kan bestyrelsen vælge at
bruge eksisterende midler og allerede indkøbt materiel (automater,
kantinens del af SAGIO-betalingssystem o.a.) til dette formål. I dette
tilfælde videreføres Kantineforeningen under de nuværende præmisser.

Hvis en flytning ikke er mulig pga. manglende passende lokaler, accept
fra institut, eller manglende midler, eller hvis bestyrelsen vurderer
at omkostningerne ved flytningen og/eller den fortsatte drift ikke
står mål med kantinens midler (både mht.  arbejdskraft og økonomi),
kan bestyrelsen vælge at afvikle kantinens aktiviteter, se næste
punkt.

\subsection*{Afvikling af kantinens aktiviteter}

Ved afvikling af kantineforeningens vil følgende gælde:

\begin{enumerate}

\item Den siddende bestyrelse pålægges at afvikle de af kantinens
fysiske aktiver, som der ikke længere vil være brug for, eller
mulighed for øko\-nomisk at opbevare (automater o.a.) således at
foreningens likvider maksimeres.

\item Foreningens formål ændres således at hensigten fremover vil være
(uprioriteret)

\begin{itemize}

\item Reetablere en studenterkantine for datalogistuderende på KU.

\item Støtte DIKU-studerendes mulighed for indtag af billig fast og
flydende føde.

\item Støtte sociale formål på DIKU - f.eks. DIKURevy, film, rejser,
fester, o.a.

\item Skulle det ikke længere være muligt at støtte sociale formål ved
datalogistudiet ved KU (f.eks. ved nedlæggelse af studiet eller hvis
studiet splittes i sådan grad at den siddende bestyrelse vurderer at
der ikke længere findes grundlag for social aktivitet for
datalogistuderende), omdannes kantineforeningen til en alumneforening
for forhenværende datalogistuderende

\end{itemize}

\item Bestyrelsen for den reformerede forening vil fremover have en
væsentlig mindre arbejdsbyrde. Bestyrelsens antal sættes derfor til
maksimalt 9 og mindst 5. Ved overgang til nedlukningsfase forbliver
nuværende med\-lemmer af bestyrelsen på pladsen, indtil deres
valgperiode udløber. Plad\-ser på valg udfyldes efter valg på
generalforsamling med, på tidspunktet for generalforsamlingen,
nuværende medlemmer af kantineforeningen.

\item Valgperioden er på 2 år.

\item Bestyrelsen kan mødes efter nødvendighed, dog mindst 2 gange om
året, og skal behandle indkomne ønsker om støtte. Det står bestyrelsen
frit selv at indsende forslag.

\item Bestyrelsen skal forbruge et beløb svarende til mindst 4\% og
maksimalt 10\% af foreningens kapital ved indstiftelsen per år.

\begin{itemize}

\item I tilfælde af at bestyrelsen ønsker at bruge et større beløb end
maksimalkravene i punkt 5, f.eks. til reetablering af kantine, skal
der indkaldes til ekstraordinær generalforsamling.

\item Bestyrelsen har diskretion til forbrug af passende beløb til
opret\-holdelse af foreningens drift.

\end{itemize}

\item Bestyrelsen har krav på dokumentation, for at verificere at
Kantinefor\-eningens midler er brugt til det formål angivet i
støtteansøgninger.

\item Bestyrelsen aflønnes ikke.

\end{enumerate}

\vspace{\fill}

\noindent \emph{Ovenstående vedtægter er vedtaget den 20. marts 1974
på den stiftende generalforsamling og ændret 25.  februar 1975, 2.
marts 1976, 14. december 1978, 24. november 1986, 30.  oktober 1990,
2.  februar 1995, 2. marts 1998, medio september 2001, 8. marts 2007,
22. april 2009, samt 6. juni 2012.}

\bigskip

\noindent Således vedtaget \vedtagsdato{}.

\end{document}
